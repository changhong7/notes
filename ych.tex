\documentclass{article}
\usepackage{ctex}

\usepackage{amsmath}
\usepackage{amsfonts}
\usepackage{amssymb}
\usepackage{amsthm}
\usepackage{bm}

\usepackage{fancyhdr}
\usepackage[top=1in, bottom=1in, left=1.25in, right=1.25in]{geometry}
\usepackage{graphicx}
\usepackage{float}
\usepackage{subfig}%次标题宏包
\usepackage{multirow}%multirow命令
\usepackage{caption}
\captionsetup[figure]{name=}
\captionsetup[table]{name=}
%%为图片和表格自动编号,name=后的东西即为表头
\usepackage{blindtext}

\usepackage{xcolor}
\newcommand{\red}[1]{\textcolor[rgb]{1.00,0.00,0.00}{#1}}
\newcommand{\blue}[1]{\textcolor[rgb]{0.00,0.00,1.00}{#1}}
\newcommand{\green}[1]{\textcolor[rgb]{0.00,1.00,0.00}{#1}}
\newcommand{\darkblue}[1]
{\textcolor[rgb]{0.00,0.00,0.50}{#1}}
\newcommand{\darkgreen}[1]
{\textcolor[rgb]{0.00,0.37,0.00}{#1}}
\newcommand{\darkred}[1]{\textcolor[rgb]{0.60,0.00,0.00}{#1}}
\newcommand{\brown}[1]{\textcolor[rgb]{0.50,0.30,0.00}{#1}}
\newcommand{\purple}[1]{\textcolor[rgb]{0.50,0.00,0.50}{#1}}
%%在文档中改变字体的颜色

%\special{papersize=8in,4.5in}
%\usepackage{syntonly}
%\syntaxonly

\newcommand{\myline}{\rule[-.4pt]{5em}{.4pt}}
%自定义空白下划线
\begin{document}

%%标题页
\author{袁长红 \and 赛小息}
\title{材料科学基础\\---材料科学基础笔记}
\date{\today}
%\CTEXoptions[today=big]
%\CTEXoptions[today=old]
\maketitle


%%页眉 页脚
\pagestyle{fancy}
\rhead{}
\lhead{}
\chead{学无止境\qquad 气有浩然}
\lfoot{}
\cfoot{\thepage}
\rfoot{}
%控制页眉页脚横线宽度
\renewcommand{\headrulewidth}{0.4pt}
\renewcommand{\footrulewidth}{0.4pt}

%%定义
\renewcommand{\contentsname}{目录}%中文目录
\def\abstractname{简述}
\newtheorem{law}{定理}
\newtheorem{definition}{定义}
\newtheorem{test}{test}
\tableofcontents


大写罗马数字1\uppercase\expandafter{\romannumeral1}

材料科学与工程学院\linebreak

下面哪个不对\hfill (\quad)\\
下面哪个不对\dotfill (\quad)\\
下面哪个不对\hrulefill (\quad)\\

{\em 文本效果}
	\underline{文本效果}
	\red{文本效果}
	\colorbox{yellow}{文本效果}
	\colorbox{yellow}{\red{文本效果么}}

\newpage
%%表格
\centering\begin{tabular}{|c|c|}
\hline
\multicolumn{2}{|c|}{Ene}\\%列数 对齐方式 文本
\hline
Mene & Muh! \\
\hline
\end{tabular}

\begin{tabular}{|c|c|}\hline
\multirow{2}*{Common text}%行数 长度 *表默认长度 文本
& Column g2a \\ \cline{2-2}
& Column g2b \\ \hline
\end{tabular}


\begin{center}
\begin{tabular}{|c|c|c|c|c|c|c|c|}\hline
\multirow{2}*{题号} & \multicolumn{5}{c|}{第一部分} &
\multirow{2}*{第二部分} & \multirow{2}*{总分}\\\cline{2-6}
 &一 ˜ & 二 &三 & 四&五 & &\\\hline
得分& & & & & & &\\ \hline
\end{tabular}
\end{center}

\newpage
%%数学公式
\flushleft
\(\sum\limits_{i=1}^{n}\)\\
$\iint \qquad \iiint \qquad \iiiint \qquad \idotsint$\\
$$\bm{a}+\bm{b}=\bm{0}$$
$$\underbrace{ a + \overbrace{b +\cdots + y}^{123} +z }_{26}$$
指定括号大小
$(\big(\Big(\bigg(\Bigg($\quad$\}\big\}\Big\}\bigg\}\Bigg\}$\\
绝对值符号$\left\vert\dfrac{a}{b}\right\vert$\\
幻影
\begin{displaymath}
{}^{12}_{\phantom{1}6}\textrm{C} \qquad
\textrm{versus} \qquad
{}^{12}_{6}\textrm{C}
\end{displaymath}
\begin{displaymath}
\Gamma_{ij}^{\phantom{ij}k} \qquad
\textrm{versus} \qquad
\Gamma_{ij}^{k}
\end{displaymath}
\begin{equation}
\begin{aligned}
&a+b=c\\
&a+b+c+d+e=f
\end{aligned}
\end{equation}
\begin{subequations}
	\begin{equation}
	a+b=c
	\end{equation}
	\begin{equation}
	a+b+c+d+e=f
	\end{equation}
\end{subequations}
\end{document}
