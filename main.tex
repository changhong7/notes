\documentclass{article}
\usepackage[utf8]{inputenc}
\usepackage{amsmath}
\usepackage{physics}
\usepackage{ctex}
\usepackage{indentfirst}
\title{qm}
\author{Yuan Changhong }
\date{October 2021}

\begin{document}

\maketitle
\section{量子力学基本理论}
\subsection{态叠加原理和希尔伯特空间}
\begin{itemize}
    \item 施密特正交化方法构造正交归一矢
\end{itemize}
\begin{enumerate}
    \item 
    证明内积定义
\begin{equation}
    \braket{\phi}{\psi}=\int{\rm d}\vec{r}\phi^*(\vec{r})\psi(\vec{r})
\end{equation}
满足
\begin{align}
    \braket{\phi}{\psi} & =\braket{\psi}{\phi}^*\\
    \braket{\phi}{\psi_1+\psi_2} & =\braket{\phi}{\psi_1}+\braket{\phi}{\psi_2}\\
    \braket{\phi}{\lambda\psi} &=\lambda\braket{\phi}{\psi}\\
    \braket{\phi}{\phi} & \geq0\\
    \braket{\phi}{\phi} & =0\quad {\rm iff}\ket{\phi}=0
\end{align}
    \item
    证明Schwartz不等式:对于任意量子态$\ket{\varphi_1}$与$\ket{\varphi_2}$,
    \begin{equation}
    \abs{\braket{\varphi_1}{\varphi_2}}^2\leq\braket{\varphi_1}{\varphi_1}\braket{\varphi_2}{\varphi_2}
    \end{equation}
    当且仅当$\ket{\varphi_1}$与$\ket{\varphi_2}$成正比时,等号才成立.
    
    事实上,假设$\ket{\varphi_1}$与$\ket{\varphi_2}$已经给定,考虑一个右矢$\ket{\psi}$,其定义为:
    \begin{equation}
        \ket{\psi}=\ket{\varphi_1}+\lambda\ket{\varphi_2}
    \end{equation}
    式中$\lambda$是一个任意参量.不论$\lambda$如何,都有
    \begin{equation}
        \braket{\psi}{\psi}=\braket{\varphi_1}{\varphi_1}+\lambda\braket{\varphi_1}{\varphi_2}+\lambda^*\braket{\varphi_2}{\varphi_1}+\lambda\lambda^*\braket{\varphi_2}{\varphi_2}
    \end{equation}
    我们给$\lambda$选择一个数值:
    \begin{equation}
        \lambda=-\dfrac{\braket{\varphi_2}{\varphi_1}}{\braket{\varphi_2}{\varphi_2}}
    \end{equation}
    整理可得:
    \begin{equation}
        \braket{\varphi_1}{\varphi_1}-\dfrac{\braket{\varphi_1}{\varphi_2}\braket{\varphi_2}{\varphi_1}}{\braket{\varphi_2}{\varphi_2}}\geq0
    \end{equation}
    由于$\braket{\varphi_2}{\varphi_2}$是一个正数,可得到:
    \begin{equation}
         \braket{\varphi_1}{\varphi_1} \braket{\varphi_2}{\varphi_2}\geq \braket{\varphi_1}{\varphi_2} \braket{\varphi_2}{\varphi_1}
    \end{equation}
    当且仅当$\braket{\psi}{\psi}=0$时,即$\ket{\varphi_1}=-\lambda\ket{\varphi_2}$,两右矢成正比时等号才成立.
\end{enumerate}
\subsection{可观测量和厄米算符}
\begin{itemize}
    \item 厄米算符的定义和性质
    \item 投影算符的定义和性质
\end{itemize}
\subsection{基本对易关系}
基本对易关系:$[\hat{r_i},\hat{p_j}]=i\hbar\delta_{ij}$仅对直角坐标系成立。

坐标平移算符/空间平移演化算符$e^{ix\hat{p}/\hbar}$ $$e^{ix'\hat{p}/\hbar}\ket{x}=\ket{x+x'}$$
$$\braket{x}{p}=\frac{1}{\sqrt{2\pi\hbar}}e^{ipx/\hbar}$$
$$\delta(x-x')=\frac{1}{2\pi}\int_{-\infty}^{\infty}e^{ik(x-x')}\rm{d}k$$
\begin{itemize}
    \item 厄米算符完备组
    \item 不确定关系:如果两个可观测量$\hat{A}$和$\hat{B}$不对易,即$[\hat{A},\hat{B}]\neq0$,则对任意量子态$\ket{\phi}$,有
    $$\Delta A\Delta B\geq\frac{1}{2}|\bra{\phi}i[\hat{A},\hat{B}]\ket{\phi}|$$
\end{itemize}
\section{文献阅读笔记}
磁场穿透深度公式:
\begin{equation}
    \lambda=\sqrt{\dfrac{m^*}{u_0n_se^2}}
\end{equation}
$m^*$为超导电子的有效质量,$n_s$为超流密度。

零温极限下,超导体的相位刚度$\rho_{s0}$正比于超流密度$n_s$,相位刚度$\rho_{s0}$越小意味着相位涨落越强。铜氧化物的相位刚度$\rho_{s0}$与超导转变温度$T_c$之间存在强劲的标度关系。

输运是一种方法相对简单却可以有效反映各种电子态信息的测量手段。例如,可以通过测量霍尔系数得到载流子浓度,测量热电势得到费米能,测量转角磁电阻研究电子态的对称性等。

\section{Structure Orderings and Symmetry Breaking}
\begin{itemize}
    \item 短程序(Short-range order,SRO):关联函数随位置指数衰减
    \item 长程序(Long-range order,LRO):关联函数随位置渐变至非零值
    \item 准长程序(Quasi long-range order,QLRO):关联函数随位置按幂律衰减
\end{itemize}

KT相变:超导相变、Heisenberg铁磁相变不是KT相变。
\begin{itemize}
    \item $T>T_c$,single vortices;
    \item $T<T_c$,bound vortex pairs.
\end{itemize}

对于海森堡铁磁模型,在高于$T_c$时,自旋关联函数在三维情况下随位置指数衰减(SRO),
在二维情况下(XY model)按幂律衰减(QLRO)。

\subsection{Incommensurate structures 非公度结构}

$x_n=na$, $a$ denotes lattice spacing, $n=$integer. New position: $X_n=x_n+g\sin{(q_0\frac{2\pi}{a}x_n)}$.
\begin{itemize}
    \item If $q_0$ is a nonzero rational number, $X_n$ gives a commensurate structure;
    \item If $q_0$ is a irrational number, $X_n$ gives an incommensurate structure with quasi periodicity(period=$\infty$);
\end{itemize}

\subsection{Quasicrystal---Penrose tiling}

\begin{itemize}
    \item Periodic crystal: one tile;
    \item Disordered solids: infinite tiles;
    \item Quasicrystals: finite tiles
\end{itemize}

\subsection{对称性破缺}

引入序参量$\Psi$,无序态$\Psi=0$,有序态$\Psi\neq 0$。
\begin{itemize}
    \item 一级相变
    \item 二级相变
\end{itemize}

\subsection{绝热近似}

\section{元激发}
\subsection{前言}
元激发分为集体激发和个体激发。
\begin{itemize}
    \item 集体激发:玻色子,如声子、等离激元、磁振子、极化子、激子
    \item 个体激发:费米子,如准电子、准空穴
\end{itemize}

费米液体:费米子量子力学液体的简称;条件是温度足够低;准粒子和原来的粒子有着同样的自旋、电荷和动量;
\subsection{声子}
弹性常数及密度周期分布的材料或结构被称为声子晶体。声子晶体是由弹性固体周期排列在另一种固体或流体介质中形成的一种新型功能材料。弹性波在声子晶体中传播时,受其内部结构的作用,在一定频率范围(带隙)内被阻止传播,而在其他频率范围(通带)可以无损耗地传播。
\end{document}

