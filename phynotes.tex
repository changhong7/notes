\documentclass{article}

\usepackage{ctex}

\usepackage{amsmath}
\usepackage{amsfonts}
\usepackage{amssymb}
\usepackage{amsthm}

\begin{document}
\section{费米液体理论}
能带理论中采用单电子近似,认为电子-电子间相互作用不会明显影响电子行为。
也就是说,这种相互作用只会在定量上修正电子的布洛赫图像,但不改变其定性特征。
然而,在真实材料中情况并非如此,电子的动能与势能相当,电子间相互作用不可忽略。
电子的动能$T$即费米能,数量级为1eV,势能
$V\approx\dfrac{e^2}{4\pi\epsilon_0a}, a\approx 10^{-10}\mathrm{m}$,
$a$是晶格常数,估算可得$T\approx V$。

对普通金属,电子间相互作用能与费米能相当,比费米面附近的能级间距大得多,
对于这样的强相互作用系统,微扰理论失效。
朗道提出,这样一个具有强相互作用的费米子系统可以在进行准粒子等效之后,忽略相互作用,
这就是费米液体理论。这有点难以置信,
这其实是因为可以对无相互作用系统逐步加入相互作用,
使其绝热地过渡为有相互作用的系统。
并不是所有的相互作用系统都可以由无相互作用系统绝热演化而来,如Luttinger liquid。

\end{document}
