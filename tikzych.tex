\documentclass[titlepage]{article}
\usepackage{ctex}
\usepackage{amsmath}
\usepackage{amsfonts}
\usepackage{amssymb}
\usepackage{fancyhdr}
\usepackage[top=1in, bottom=1in, left=1.25in, right=1.25in]{geometry}
\usepackage{titlesec}
\usepackage{graphicx}
\usepackage{booktabs}
\begin{document}
\author{袁长红\\201700150122}
\title{固体物理作业}	
\date{\today}
\maketitle

\pagestyle{fancy}
\chead{袁长红}
1.求由5个原子组成的一维单原子晶格的振动频率。

由一维简单晶格振动的色散关系和周期性边界条件可得
\begin{align}
\omega=&2\sqrt{\dfrac{\beta}{m}}\left| \sin(\dfrac{qa}{2})\right|\label{f1}\\
q=&\dfrac{2\pi l}{Na}(l\in\mathbb{Z},-\dfrac{N}{2}<l\leqslant\dfrac{N}{2}) \label{f2}
\end{align}
由题$ N=5,-\dfrac{5}{2}<l\leqslant\dfrac{5}{2}$ ,可知$ l=-2,-1,0,1,2 $ . 将\eqref{f2}和$ N=5 $代入\eqref{f1}中得
\begin{equation}\label{f3}
\omega=2\sqrt{\dfrac{\beta}{m}}\left| \sin(\dfrac{\pi l}{5})\right|
\end{equation}
将$ l $的可能取值依次代入\eqref{f3}中,得晶格的振动频率$ \omega_1=2\sin\dfrac{2\pi}{5}\sqrt{\dfrac{\beta}{m}}\approx1.902\sqrt{\dfrac{\beta}{m}}(l=-2,2),\omega_2=2\sin\dfrac{\pi}{5}\sqrt{\dfrac{\beta}{m}}\approx1.176\sqrt{\dfrac{\beta}{m}}(l=-1,1),\omega_3=0(l=0) $.

2.晶体振动的动力学基础。



3.如何理解单个原子振动和多个振动模式?

单个原子的振动是多个振动模式(相互独立的简谐振动)的格波叠加的结果。(类似于量子力学中任意态可由多个相互正交的本征态线性组合得到)
\newpage
1. $ q=\frac{\pi}{2a} $时,在声学支格波上,质量为$ m $的轻原子保持不动;在光学支格波上,质量为$ M $的重原子保持不动。

对声学支和光学支格波,相邻原子的振幅之比为
\begin{align}
\dfrac{A}{B}=\dfrac{2\beta\cos(qa)}{2\beta-m\omega^2}\label{f3}\\
\dfrac{A}{B}=\dfrac{2\beta-M\omega^2}{2\beta\cos(qa)}\label{f4}
\end{align}
\eqref{f3}\eqref{f4}两式等价。$ q=\frac{\pi}{2a}\Rightarrow\cos(qa)=0 $,在声学支格波上$ \omega=\sqrt{\frac{2\beta}{M}} $,代入\eqref{f3}中得$ \dfrac{A}{B}=0\Rightarrow A=0 $,所以质量为$ m $的轻原子保持不动(代入\eqref{f4}中得到0/0不定式);在光学支格波上$\omega=\sqrt{\frac{2\beta}{m}}  $,代入\eqref{f4}中得$ \dfrac{A}{B}=\infty\Rightarrow B=0 $,所以质量为$ M $的重原子保持不动(代入\eqref{f3}中得0/0不定式)。

2. 一维无限长原子链,原子质量为$ m $和$ M $,且$ m<M $。靠得较近的两个原子构成一个分子。设一个分子内两原子平衡位置的距离为$ b $,恢复力系数为$ \beta_1 $,分子间两原子间的恢复力系数为$ \beta_2 $,晶格常量为$ a $(图\ref{fig:1}所示),求色散关系。
% TODO: \usepackage{graphicx} required
\begin{figure}[ht]
	\centering
	\includegraphics[width=0.7\linewidth]{fig/1}
	\caption{示意图}
	\label{fig:1}
\end{figure}

运动方程:
\begin{equation}\label{f5}
\begin{cases}
m\dfrac{\intd^2x_{2n+1}}{\intd t^2}=-\beta_2(x_{2n+1}-x_{2n+2})-\beta_1(x_{2n+1}-x_{2n})\\
M\dfrac{\intd^2x_{2n+2}}{\intd t^2}=-\beta_2(x_{2n+2}-x_{2n+1})-\beta_1(x_{2n+2}-x_{2n+3})
\end{cases}
\end{equation}
试探解:\begin{equation}\label{f6}
\begin{cases}
x_{2n+1}=A\exp [i(\omega t-qr_{2n+1})]=A\exp \{i[\omega t-q(na+b)]\}\\
x_{2n+2}=B\exp [i(\omega t-qr_{2n+2})]=B\exp \{i[\omega t-q(n+1)a]\}
\end{cases}
\end{equation}
由试探解可得
\begin{equation}\label{f9}
\begin{cases}
\dfrac{\intd^2x_{2n+1}}{\intd t^2}=-\omega^2x_{2n+1}\\
x_{2n+2}=\dfrac{B}{A}\exp[iq(b-a)]x_{2n+1}\\
x_{2n}=\dfrac{B}{A}\exp(iqb)x_{2n+1}\\
\dfrac{\intd^2x_{2n+2}}{\intd t^2}=-\omega^2x_{2n+2}\\
x_{2n+1}=\dfrac{A}{B}\exp[iq(a-b)]x_{2n+2}\\
x_{2n+3}=\exp(-iqa)x_{2n+1}
\end{cases}
\end{equation}
将试探解\eqref{f6}代入\eqref{f5}中利用\eqref{f9}整理可得
\begin{equation}\label{f7}
\begin{cases}
[m\omega^2-(\beta_1+\beta_2)]A+\{\beta_2\exp[iq(b-a)]+\beta_1\exp(iqb)\}B=0\\
\{\beta_2\exp[iq(a-b)]+\beta_1\exp(-iqb)\}A+[M\omega^2-(\beta_1+\beta_2)]B=0
\end{cases}
\end{equation}
若$ A $、$ B $有非零解,则系数行列式必须等于0,即
\begin{equation}\label{f8}
\begin{vmatrix}
m\omega^2-(\beta_1+\beta_2)& \beta_2\exp[iq(b-a)]+\beta_1\exp(iqb) \\
\beta_2\exp[iq(a-b)]+\beta_1\exp(-iqb)& M\omega^2-(\beta_1+\beta_2)
\end{vmatrix}=0
\end{equation}
行列式展开得\begin{equation}\label{f10}
mM\omega^4-(\beta_1+\beta_2)M\omega^2-(\beta_1+\beta_2)m\omega^2+(\beta_1+\beta_2)^2-[\beta_2^2+\beta_1\beta_2\exp(iqa)+\beta_1\beta_2\exp(-iqa)+\beta_1^2]=0
\end{equation}
由\eqref{f10}利用欧拉公式化简得
\begin{equation}\label{f11}
mM\omega^4-(\beta_1+\beta_2)(m+M)\omega^2+4\beta_1\beta_2\sin^2\frac{qa}{2}=0
\end{equation}
\eqref{f11}是关于$ \omega^2 $的一元二次方程,其中$ \Delta=(\beta_1+\beta_2)^2(m+M)^2-16mM\beta_1\beta_2\sin^2\dfrac{qa}{2}$
\begin{equation}\label{f12}
\omega^2=\dfrac{(\beta_1+\beta_2)(m+M)\pm\sqrt{\Delta}}{2mM}
\end{equation}
\eqref{f12}中声学支格波取负号,光学支格波取正号。由\eqref{f12}知$ \omega_{A/O}-q $的周期为$ \dfrac{2\pi}{a} $,所以可将$ q $限制在$(-\dfrac{\pi}{a},\dfrac{\pi}{a}]$。根据周期性边界条件(设晶体有$ N $个原胞)有\begin{equation}\label{key}
x_{2n+1}=x_{2n+1+2N}
\end{equation}
得$ \exp(iqNa)=1 $,解得$ qNa=2\pi l(l\in\mathbb{Z})\Rightarrow q=\dfrac{2\pi l}{Na}$,又$ q\in(-\dfrac{\pi}{a},\dfrac{\pi}{a}]\Rightarrow l\in(-\dfrac{N}{2},\dfrac{N}{2}] $。

3.
\begin{table}[ht]
	\centering
	\begin{tabular}{@{}cccccccc@{}}
		\toprule
		& 原子数/原胞 & 原胞数 & 自由度数 & 波矢数 & 格波数\footnotemark & 声学波数(支) & 光学波数(支) \\ \midrule
		单原子链 & 1      & $N$ &    $ N $  &   $ N $  & $ N $    &     1    &     0    \\
		双原子链 & 2      & $N$ &      $ 2N $&   $ N $  &  $ 2N $   &    1     &    1     \\
		三维结构 & $n$    & $N$ &      $ 3nN $&   $ N $  &   $ 3nN $  &   3      &   $ 3(n-1) $      \\ \bottomrule
	\end{tabular}

\end{table}
\footnotetext{晶体振动模式或$ (\omega,q)$数}

\end{document}



