\documentclass{article}

\usepackage{ctex}
\usepackage{amsmath}
\usepackage{amsfonts}
\usepackage{amssymb}
\usepackage{amsthm}
\usepackage{bm}
\usepackage[top=1in, bottom=1in, left=1.25in, right=1.25in]{geometry}

\usepackage[version=4]{mhchem}%化学式
\begin{document}

\section{凝聚态物理的连续革命}

固体物理的杰出贡献之一是对金属物理学的普适性理解,这基于20世纪50年代和60年代的三大发展:
\begin{itemize}
    \item 基于重整化的多体技术,
    \item BCS超导理论,
    \item 能带理论。
\end{itemize}

重整化是量子电动力学中使用的一种方法,其对电子的质量和电荷等性质的数值进行修改,
以考虑与真空涨落的相互作用。
理论物理学家对金属中费米面附近的电子进行了重整化。
在金属中,来自每个原子的一个或几个电子可以自由移动,
由此产生的电子气密度非常大,以至于泡利不相容原理迫使其平均动能比热能大。
尽管如此,这种强相互作用的量子态在某种意义上可以被处理为粒子之间没有相互作用:
其他电子的稠密气体在修正单粒子性质时被“重整化”。
这种重整化的电子或准粒子控制着金属的整个动力学和输运性质。
此外,这种近似在低能极限下是精确的。

BCS超导理论是由约翰·巴丁、利昂·库珀和罗伯特·施里弗于1957年以一种简化的形式提出的,
除了一种非常特殊的电子之间的相互作用外,它忽略了所有其他方面。
BCS理论在解释超导现象学(电阻损耗、比热、磁通量量子化、约瑟夫森效应等)方面相当令人满意。

能带理论的发展包括詹姆斯·菲利普斯、沃尔克·海涅等人的赝势理论,
以及约翰·斯莱特的局域密度近似LDA。
其他许多人利用这些来计算准粒子运动的能带。
“赝势”是一种避免在计算中包含所有内壳层电子的方法。
它们还削弱和简化了导致能带形成的电子-离子相互作用。 

除了成功之外,固体物理也遭遇了一些困难。
20世纪50年代,Sin-Itoro Tomonaga和Joaquin Luttinger证明,
假设的一维金属最好用集体激发(例如密度波和孤子)来描述,
而不是用准粒子(此处特指个体激发的准粒子)来描述,尽管这些激发通常有自己的费米面。
另一个困难是二维量子霍尔效应——包括整数和分数量子霍尔效应。
该效应主要由朗道能级之间的能隙决定,这种能隙完全使标准微扰方法无效,
但可以用分数荷电孤子进行处理。

然而,“普通”条件下的“普通”金属似乎完全可以用费米液体理论来解释。
费米液体理论甚至还取得了一些意想不到的成功。
其中之一就是Nozi$\grave{e}$res在安德森和同事以及肯·威尔逊早期工作的基础上认识到,
近藤效应实质上将磁性杂质重整化为费米液体基态。
(近藤效应是杂质磁矩和传导电子自旋之间的反铁磁相互作用,由近藤于1963年提出)
近藤效应似乎也解释了一组含有铈、铀、钐或镱的金属化合物的特殊行为,即量子价波动金属。
这些金属是不寻常的,因为在室温下,f壳层中的电子更喜欢在不同的价态之间隧穿,
而不是移动。但在低于10-100K的温度下,它们变得可移动,
如果很重的话,会增加费米表面的体积。

1987年铜氧化物高温超导体的发现促使人们重新评估费米液体理论。
在超导转变温度$T_c$以上,\ce{CuO2}平面中的“正常”金属的行为不像费米液体。
随后,人们提出了多种液体理论,
如Anderson的液体(二维气体类似于一维Luttinger液体的流形,费米面的每一点对应一种)、
Laughlin的"anyon"液体、
Chandra Varma和Peter littlewood等人的边缘费米液体。
边缘费米液体作为一种经验假设,对费米液体的改动最小。
实验数据压倒性地支持了Anderson的理论。

Luttinger液体是一维问题许多解的一个共同特征。
这些液体有费米面,满足Luttinger定理,即费米表面的模体积是常数。
然而,在费米液体中,仅仅是表面的不连续性,
而在Luttinger液体中,只有幂律奇点。
Luttinger液体是几种已知或假定的非费米液体之一,可能是最极端的
(除了“量子霍尔液体”,它没有费米表面)。

如果像高温超导体这样的二维金属的常见的例子都不是费米液体,
那么在其他情况下能在多大程度上依赖传统的费米液体理论呢?
总的来说,费米液体理论有两个漏洞。
第一点是实验性上的:它一直未能解释关于金属的一系列实验事实,
在这些事实中,由于某种原因,电子之间的相互作用预计会很强。
这包括了像\ce{NbSe2}和\ce{TaS2}等层状二卤化物、
\ce{V3Si}和\ce{PbMo6S8}等不寻常的超导体、
重费米子超导体和有机超导体。
第二点是理论上的:
当来自费米能级附近的电子从金属的导带落入内壳层时,
X射线的光谱显示出对电子能量的反常幂律依赖性,而不是费米统计预测的尖锐边缘。 

物理学家们花费了大量的时间试图将非费米液体行为纳入费米液体理论的框架之中,
但没有成功,或许答案在于要构建全新的非费米液体理论。

\section{强关联电子系统}
能带理论在解释固体中的电子性质和现象方面取得了许多重大成功(至少在定性上),
这似乎表明电子之间的相互作用往往只会在定量上改变布洛赫图像,而不会改变其定性特征。
然而,在强关联电子系统中,能带理论中的单电子近似完全失效,出现了不少新奇的现象,
比如重费米子、量子霍尔效应、高温超导等(依次简要介绍)。
对此,物理学家们提出了重整化、准粒子、对称性破缺这三个新概念来理解电子行为。


固体材料由大量粒子(数量级$10^{23}$,如电子、离子)组成,
粒子之间存在着很强的相互作用,是一个复杂的多体系统,严格求解十分困难。
元激发这一概念是在研究固体的低激发态时引入的。
能量靠近基态的低激发态与其他激发态相比较,情况较为简单:
这种低激发态往往可以看成一些独立的基本激发单元的集合。
这些基本激发单元称为元激发或准粒子。
准粒子具有确定的能量,有时还有确定的准动量(如声子)。
准粒子概念的引入,可以将复杂的多体系统简化成接近于理想气体的准粒子系统。
准粒子可以分为两类:
\begin{itemize}
    \item 集体激发的准粒子。
    晶格振动的格波---声子、自旋波---磁振子、等离子区集体振荡---等离激元
    \item 个体激发的准粒子。
\end{itemize}


对称性破缺是一个用来描述宏观系统的术语,
这些宏观系统自发地发展出破坏其周围环境对称性的组织模式。
例如,在晶体中,原子形成了一种均匀有序的结构,打破了平移对称性。
在磁铁中,电子排列自旋以打破旋转对称性。
当一个系统打破对称性时,它就会形成一种刚度。
刚度导致新型力的传递,可将其与“超导电流”联系起来。
在超导体中,刚度在于电子对的量子力学相位。
电子相位的梯度产生了一种新的力,这就是超导电流。

\subsection{重费米子}

20世纪70年代末首次发现了重费米子金属。
这一发现迫使凝聚态物理学家修正了对局域磁矩与电子流体相互作用的理解。
70年代末,金属中电子的行为被划分为:
\begin{itemize}
    \item 离域行为,电子形成布洛赫波;
    \item 定域行为,电子束缚在材料中某个特定原子附近。
    这种未配对的电子自旋形成了被称为“局域磁矩”的微小原子磁体,
    在低温下倾向于排列,对超导性有极大的破坏。
\end{itemize}

重费米子金属完全不符合这两种规范,它含有密集的磁性稀土或锕系元素原子排列矩阵,
这些原子在高温下随机取向。
冷却时,磁矩与周围的电子形成高度相关的顺磁基态,而不是有序地排列成刚性的磁性结构。
当这种情况发生时,金属的电阻率突然下降,形成高度相关的费米液体。
不严格地说,每一个局域矩都会被一个自旋相等且相反的电子云磁屏蔽。
这种新的准粒子激发类似于自由电子,但它们有效质量很大,比传统金属的质量大1000倍。 

这种重费米子流体通常会进一步发展形成新奇的磁铁、超导体和绝缘体。
重费米子在绝缘体中形成的带隙可以比硅小几百倍;
重费米子超导体也比较特殊,磁矩与超导性共存,但在传统金属中,磁矩会破坏超导性。

重费米子行为在金属的比热容中有着明显的体现。
在传统金属中,激发的准粒子数量随温度线性增长,这导致电子比热正比于温度,
比例系数正比于电子的有效质量;
由于重费米子的有效质量大得多,
重费米子金属的比热远大于传统金属的比热。

20世纪50年代,朗道证明了20世纪30年代发展起来的费米液体的概念可以推广。
他认为,如果电子之间的相互作用能够以平滑、绝热的方式开启,
那么低能激发将保留自由粒子的性质,但具有“重整化”性质和增强的质量。
重费米子正是这种重整化准粒子的极端例子。 

\subsection{量子霍尔效应}
量子霍尔效应的发现是20世纪凝聚态物理学的一项辉煌成就,
两届诺贝尔物理学奖颁发给了整数和分数量子霍尔效应的发现者或研究者。
量子霍尔效应的理论,涉及现代物理学的许多基本概念,
例如基态、激发态、元激发及其分数统计 、对称性破缺等。
20世纪30年代,朗道讨论了磁场中电子运动的量子力学,
预测了电子的能量会量子化为离散的、等间距的朗道能级$E_n$。
\begin{equation*}
    E_n=(n+\frac{1}{2})\hbar\omega_c,\quad n=1,2,3\dots,\quad\omega_c=eB/m
\end{equation*}
Landau量子化已在金属中得到证实,这会导致
金属的电阻率和磁化率随磁场的变化而产生振荡。
在70年代,半导体技术的进步(异质结)和高磁场的可用性,
使得在高磁场下研究二维电子体系成为可能,此时朗道能级的间距非常大,
以至于所有电子落入最低朗道能级(抑制了动能),因此它们的运动完全由电子之间的库仑相互作用控制。

1980年初,德国物理学家冯·克利青在二维体系的霍尔效应实验中,
发现了与经典霍尔效应完全不同的现象:
MOSFET(金属—氧化物半导体场效应晶体管)的霍尔电阻并不随磁场强度的增大而线性增加,
其霍尔电阻$R_H$是量子化的(台阶式变化,出现霍尔电阻平台;
霍尔电阻处于平台时,纵向电阻为零),
\begin{equation*}
    R_H=\frac{h}{\nu e^2}
\end{equation*}
其中,$e$是电子电荷量的绝对值,$h$是普朗克常数。
在较低的磁场下,$\nu=1,2,3\dots$,是正整数,称为填充因子,
它是由电子密度和磁通密度决定的,这就是整数量子霍尔效应。
整数量子霍尔效应的物理机制可由朗道能级和边缘态予以解释:
\begin{itemize}
    \item 按经典电磁理论,处于磁场中的电子因受洛伦兹力作圆周运动形成回旋轨道。
    因此,样品中间的电子被磁场“局域化”在原地打转,使样品中间成为绝缘体。
    在量子力学中,根据朗道能级,电子的轨道是量子化的。
    某一时刻,若电子按泡利不相容原理共填充了$\nu$(正整数)个朗道能级,
    此时得到的霍尔电阻平台,正好对应$R_H=\frac{h}{\nu e^2}$; 
    \item 霍尔电阻处于某一平台值时,样品无纵向电阻:
    在样品边缘的电子,虽然受洛伦兹力而作圆周运动,但还未完成一个圆周运动就被边缘反弹回来;
    继续圆周运动,继续反弹,如此反复,电子就可沿样品的边缘从一端运动到另一端,形成“边缘态”。
    由于外加磁场的作用,电子只能沿着一个方向运动。边缘的电子当碰到杂质时,会绕一个弯继续向前,
    不会发生散射。散射是电子能量发生耗散形成电阻的主要原因,
    所以在量子霍尔态电子的运动是无能量损耗的,即电流流过样品而无纵向电阻。
\end{itemize} 

但在更高的磁场中,
$\nu=\frac{1}{3},\frac{1}{5},\frac{1}{7}\dots$,是分数,称为分数量子霍尔效应。
1983年,Laughlin对该效应作出了解释,其要点为:
\begin{itemize}
    \item 分数量子霍尔效应是一种低维电子系统的强关联效应,由电子间的库仑排斥相互作用所致
     (整数量子霍尔效应不考虑电子关联)
    \item 给出了电子系统的基态波函数,被称为劳克林波函数,用波函数计算系统基态的能量
    \item 分数量子平台是多电子系统集合态的特征,称为量子霍尔液体。
    此液体用消耗能量作为代价,产生“准粒子”,每一个准粒子携有分数电荷。
    此液体是不可压缩的,电子间的库仑相互作用为这种不可压缩性提供了能隙
\end{itemize}
这一突破开启了整个凝聚态物理研究领域进入强关联电子的新世界,
带来了一系列新概念和语言,比如
\begin{itemize}
    \item 分数统计准粒子
    \item 复合费米子
    \item Chern-Simons项
\end{itemize}
另外,分数量子霍尔效应使人们深刻地意识到,当相互作用的强度与动能相当或大于动能时,
物质中的电子可能发生深刻的变化。

\subsection{高温超导}
1986年,人们发现了铜氧化物高温超导体,其超导转变温度超过了理论上预测的麦克米兰极限(40K),
这使物理学界震惊。
目前常压下的最高超导转变温度为165K。
铜氧化物高温超导材料包含夹在绝缘氧化物层之间的氧化铜金属层,将带电载流子提供给导电平面。
高温超导体系是通过向绝缘体中掺杂电子或空穴形成的,
而在未掺杂的绝缘体中,电子局域在反铁磁序中。
高温超导材料从根本上挑战了物理学家对相关电子系统的理解,尤其是:
\begin{itemize}
    \item 相图中绝缘和超导行为之间的密切关系,
    这表明绝缘体和超导体可能来自密切相关的基态波函数
    \item 最佳掺杂材料的奇异金属行为(非费米液体),如线性电阻率
    (传统金属中电阻率随温度增加二次增长)
    \item 欠掺杂超导体电子光谱中赝能隙的起源
    (可能是在超导形成之前,电子对已经形成,但没有足够的相干性)
 \end{itemize}

 高温超导相图的研究孕育了许多新的物理思想,比如
 \begin{itemize}
     \item 自旋---电荷分离(自旋---电荷耦合电子分裂为独立的集体电荷和自旋激发)
     \item 量子临界性、量子相变
     \item 电子预配对
     \item 共振价键理论
     (超导性可以看作是一种由无自旋带电空穴构成的流体,
     在单态自旋对的背景下运动。)
     \item 规范理论的新形式,包括$Z_2$、$SU$甚至超对称规范理论,
     这些理论可能描述了掺杂莫特绝缘体中电子之间的强库仑相互作用高度约束的多种状态
 \end{itemize}


\end{document}