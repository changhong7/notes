\documentclass{article}

\usepackage{ctex}
\usepackage{amsmath}
\usepackage{amsfonts}
\usepackage{amssymb}
\usepackage{amsthm}
\usepackage{bm}
\usepackage[top=1in, bottom=1in, left=1.25in, right=1.25in]{geometry}
\begin{document}
对称性破缺、序参量、平均场理论、临界现象、元激发、费米液体

固体物理的杰出贡献之一是对金属物理学的普适性理解,这基于20世纪50年代和60年代的三大发展:
\begin{itemize}
    \item 基于重整化的多体技术,
    \item BCS超导理论,
    \item 能带理论。
\end{itemize}

重整化是量子电动力学中使用的一种方法,其对电子的质量和电荷等性质的数值进行修改,
以考虑与真空涨落的相互作用。
理论物理学家对金属中费米面附近的电子进行了重整化。
在金属中,来自每个原子的一个或几个电子可以自由移动,
由此产生的电子气密度非常大,以至于泡利不相容原理迫使其平均动能比热能大。
尽管如此,这种强相互作用的量子态在某种意义上可以被处理为粒子之间没有相互作用:
其他电子的稠密气体在修正单粒子性质时被“重整化”。
这种重整化的电子或准粒子控制着金属的整个动力学和输运性质。
此外,这种近似在低能极限下是精确的。

BCS超导理论是由约翰·巴丁、利昂·库珀和罗伯特·施里弗于1957年以一种简化的形式提出的,
除了一种非常特殊的电子之间的相互作用外,它忽略了所有其他方面。
BCS理论在解释超导现象学(电阻损耗、比热、磁通量量子化、约瑟夫森效应等)方面相当令人满意。

能带理论的发展包括詹姆斯·菲利普斯、沃尔克·海涅等人的赝势理论,
以及约翰·斯莱特的局域密度近似LDA。其他许多人利用这些来计算准粒子运动的能带。
“赝势”是一种避免在计算中包含所有内壳层电子的方法。
它们还削弱和简化了导致能带形成的电子-离子相互作用。 


\section{费米液体}

\section{准粒子}
固体材料由大量粒子(数量级$10^{23}$,如电子、离子)组成,
粒子之间存在着很强的相互作用,是一个复杂的多体系统,严格求解十分困难。
元激发这一概念是在研究固体的低激发态时引入的。
能量靠近基态的低激发态与其他激发态相比较,情况较为简单:
这种低激发态往往可以看成一些独立的基本激发单元的集合。
这些基本激发单元称为元激发或准粒子。
准粒子具有确定的能量,有时还有确定的准动量(如声子)。
准粒子概念的引入,可以将复杂的多体系统简化成接近于理想气体的准粒子系统

准粒子可以分为两类:
\begin{itemize}
    \item 集体激发的准粒子。
    晶格振动的格波---声子、自旋波---磁振子、等离子区集体振荡---等离激元
    \item 个体激发的准粒子。
\end{itemize}


\section{量子霍尔效应}

\section{高温超导}


\end{document}