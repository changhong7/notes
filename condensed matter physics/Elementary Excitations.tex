\section{Elementary Excitations}

\subsection{前言}
元激发分为集体激发和个体激发。
\begin{itemize}
    \item 集体激发:玻色子,如声子、等离激元、磁振子、极化子、激子
    \item 个体激发:费米子,如准电子、准空穴
\end{itemize}

费米液体:费米子量子力学液体的简称;条件是温度足够低;准粒子和原来的粒子有着同样的自旋、电荷和动量

边缘费米液体

Luttinger liquid

\subsection{声子}

弹性常数及密度周期分布的材料或结构被称为声子晶体。
声子晶体是由弹性固体周期排列在另一种固体或流体介质中形成的一种新型功能材料。
弹性波在声子晶体中传播时,受其内部结构的作用,在一定频率范围(带隙)内被阻止传播,
而在其他频率范围(通带)可以无损耗地传播。

\subsection{等离激元和准电子}

\subsection{激子}

激子是由受激电子和空穴由于库仑作用形成的束缚对,
存在于绝缘体、本征半导体和一些液体中。
\begin{itemize}
    \item Wannier-Mott exciton:激子半径远大于晶格常数,激子中的电子和空穴弱束缚
    \item Frenkel excition:激子半径小于等于晶格常数,激子中的电子和空穴强束缚
\end{itemize}

\subsection{极化子}

极化子是费米子。

\subsection{极化激元}

极化激元(电磁激元)是指光子与其他粒子或准粒子(例如等离激元、声子、激子等)发生强耦合后形成的玻色子。
