\section{Magnetism of Matter}

\subsection{原子的磁性}

\subsubsection{多电子原子(离子)的电子状态}

多电子原子所处的电子状态决定了原子的磁性。
原子中内部的满壳层角动量和磁矩都等于零,
因而原子的电子状态主要取决于比较靠外面的不满壳层。
把原子核和内部满壳层一起看成离子实,
讨论不满壳层的电子在离子实的势场中的运动。
哈密顿量为
$$
H=\sum_{i} H_{i}^{(0)}+\sum_{i<j} \frac{q^{2}}{4 \pi \varepsilon_{0} r_{i j}}+\sum_{i} H_{i}^{S O}
$$
第一项表示单电子哈密顿量
$$
H_{i}^{(0)}=\frac{\bm{p}_{i}^{2}}{2 m}+V(r_{i})
$$
包括电子动能和离子实的势能$V(r_{i})$,第二项表示电子之间的库仑相互作用,
第三项为自旋-轨道耦合项。

\begin{itemize}
    \item L-S耦合:电子间库仑相互作用大于自旋-轨道相互作用
    \item J-J耦合:电子间库仑相互作用小于自旋-轨道相互作用
\end{itemize}
对于不太重的元素,都属于L-S耦合。

\subsubsection{Hund's rules}
